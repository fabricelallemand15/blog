\documentclass[border=10pt]{standalone}
\usepackage{tikz}
\usepackage{tikz-qtree}
\usepackage{amsmath} % Nécessaire pour la commande \bar

\begin{document}

\begin{tikzpicture}
% Définition des styles pour les niveaux et les étiquettes
\tikzset{
  grow'=right,
  level distance=4cm, % Distance entre les niveaux
  % Style général pour les arêtes (flèches)
  edge from parent/.style={draw, thick, ->},
  % Style pour le premier niveau de l'arbre
  level 1/.style={
    sibling distance=3cm
  },
  % Style pour le deuxième niveau de l'arbre
  level 2/.style={
    sibling distance=3.5cm, 
  },
  % Style personnalisé pour les étiquettes de probabilité (boîte blanche)
  prob/.style={
    fill=white,       % Fond blanc
    inner sep=2pt,    % Marge interne
    midway            % Positionné au milieu de la branche
  }
}

% Construction de l'arbre
\Tree
  [.{ }
    % Première branche principale
    \edge node[prob, above] {$P(A)=0,6$} ;
    [.{A}
        % Sous-branches de A
        \edge node[prob, above] {$P_A(B)=0,3$} ;
        [.{B} ]
        \edge node[prob, below] {$P_A(\overline{B})=0,7$}; % Corrigé de l'image
        [.{$\overline{B}$} ]
    ]
    % Deuxième branche principale
    \edge node[prob, below] {$P(\overline{A})=0,4$} ;
    [.{$\overline{A}$}
        % Sous-branches de A-barre
        \edge node[prob, above] {$P_{\overline{A}}(B)=0,4$} ;
        [.{B} ]
        \edge node[prob, below] {$P_{\overline{A}}(\overline{B})=0,6$} ;
        [.{$\overline{B}$} ]
    ]
  ]

\end{tikzpicture}

\end{document}